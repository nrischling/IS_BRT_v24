
\documentclass{article}
\usepackage{tikz-cd}
\usepackage{amsmath}
\usepackage{amssymb}
\usepackage{amsfonts}
\usepackage{authblk}
\begin{document}
\title{Biomimetic Replicant Theory: Generative Fractal Games}
\author{Nathan S. Rischling}
\affil{Columbia University}
\maketitle
%\hyperlink{mailto:nsr2143@columbia.edu}}%[nsr2143@columbia.edu]{mailto:nsr2143@columbia.edu}}






%\begin{tikzcd}[column sep=large, row sep=huge]
%& \text{Gödel's Incompleteness} \arrow[d, "Inspired %questioning" description] & \text{Heisenberg's Uncertainty} %\arrow[ld, "Highlighted limitations" description] \\
%\text{Game Theory} \arrow[r, "Initiated exploration" %description] & \text{Digital Twins} \arrow[d, "Synthesized concepts" description] & \text{Fractals \& Scaling} \arrow[l, "Suggested self-similarity" description] \arrow[ld, "Provided insight" description] \\
%& \text{BRT} \arrow[d, "Found applications" description] & \text{Constructor Theory} \arrow[ld, "Extended ideas" description] \\
%& \text{Knowledge Mgmt \& Org. Design} & \text{Category Theory} \arrow[l, "Provided framework" description] 
%\end{tikzcd}
\begin{center}
\begin{tikzcd}
\text{Player A} \arrow[r, "f", shift left] & \text{Combined Player} \arrow[l, "g", shift left] \arrow[loop right, "h"] \\
\text{Player B} \arrow[u, "i", shift left] \arrow[ur, "j", swap] & \text{Game} \arrow[u, "k"] \arrow[l, "l", swap]
\end{tikzcd}
\end{center}
\section*{Key for Generative Fractal Game (GFG) Diagram}

\begin{itemize}
    \item \textbf{Players}: Entities or units that participate in the game. They can be individual agents, teams, companies, species, or any defined set of participants.
    
    \item \textbf{GFG (Generative Fractal Game)}: Represents the overarching game structure in which all players operate. It's a dynamic, ever-evolving game where strategies and equilibria can shift over time. 
    
    \item \textbf{Dominant/Dominated Strategies}: These are strategies that, when adopted, yield the best outcome for a player given the strategies chosen by other players. Dominant strategies always yield the best outcome while dominated strategies are sub-optimal.
    
    \item \textbf{Equilibrium}: A state where players have settled on strategies such that no player can benefit from changing their strategy while the other players keep theirs unchanged.
    
    \item \textbf{Dynamic Evolution of Strategies}: Indicates that over time, the strategies that players adopt can change, leading to new equilibria.
    
    \item \textbf{Additive Composition}: Reflects the notion that individual players can come together to form a larger, composite player.
    
    \item \textbf{Arrows}: Indicate relationships, transitions, or influences between the defined components. For instance, the arrow from Players to GFG indicates that players form and participate in the GFG.
\end{itemize}

\bigskip

By integrating Nash's equilibrium concept with the fractal, generative, and incompleteness aspects of GFGs, we can conceptualize a multi-scalar, dynamic system where equilibria at smaller scales contribute to, and are influenced by, equilibria at larger scales. This interplay between scales, the shifting nature of equilibria, and the ever-evolving game dynamics offers a rich tapestry to understand complex systems, whether they be ecological, economic, or organizational.


\begin{tikzcd}
\text{Bee Species} \arrow[r, "Pollinates"] \arrow[d, "Gathers knowledge from"']
& \text{Flowers} \arrow[d, "Transfers knowledge to"] \\
\text{Manager A} \arrow[r, "Seeks advice from"]
& \text{Manager B}
\end{tikzcd}

\end{document}

