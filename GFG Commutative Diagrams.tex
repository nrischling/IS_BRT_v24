\documentclass{article}
\usepackage{tikz-cd}
\usepackage{amssymb}
\usepackage{amsmath}

\begin{document}

\begin{center}
\begin{tikzcd}[cells={nodes={draw=gray, ultra thin, dashed}}, column sep=small]
& \star \arrow[ld, "Influence" description] \arrow[rd, "Influence" description] & \\
\star & & \star \\
& \begin{minipage}{0.6\textwidth}
\textbf{Key:} \\
\text{\textbullet} \ \text{Nodes (Stars): Players or systems in the GFG} \\
\text{\textbullet} \ \text{Arrows: Influence or interaction between players} \\
\text{\textbullet} \ \text{Nested Structures: Multi-scalar nature of GFG} \\
\text{\textbullet} \ \text{Star Symbols: Equilibrium states within system} \\
\text{\textbullet} \ \text{Incomplete Boundaries: Gödel's incompleteness}
\end{minipage} & 
\end{tikzcd}
\end{center}

\bigskip

\noindent By integrating Nash's equilibrium concept with the fractal, generative, and incompleteness aspects of GFGs, we can conceptualize a multi-scalar, dynamic system where equilibria at smaller scales contribute to, and are influenced by, equilibria at larger scales. This interplay between scales, the shifting nature of equilibria, and the ever-evolving game dynamics offers a rich tapestry to understand complex systems, whether they be ecological, economic, or organizational.

\end{document}
